La particularit\'e des graphes Iourtes est que chaque sommet et son voisinage ne peuvent \^etre partitionn\'es en deux ensembles tels que chaque ensemble induit une clique.
Cela implique que tous les sommets de ces graphes appartiennent \`a l'ensemble $\cal C$ des sommets \`a corriger.
\newline
Dans la section \ref{graphesIourtesChapitreLinegraphe}, nous avons montr\'e que les line-graphes issues de la correction des graphes Iourtes ont leurs distances line qui poss\`edent une borne sup\'erieure.
Nous allons v\'erifier si la distance line des line-graphes propos\'es converge vers cette borne sup\'erieure. Nous corrigeons alors les sommets de $\cal C$ avec le mode {\em al\'eatoire sans remise}  et nous utilisons diff\'erentes fonctions de co\^uts.
\newline
Nous d\'ebutons notre analyse par la description du protocole d'exp\'erimentation. Ensuite nous pr\'esentons les r\'esultats obtenus avec chaque fonction de co\^uts et enfin nous d\'eterminons la meilleure fonction de co\^ut. 