% conclusion generale 
Le chapitre \ref{chapitreEvaluation} analyse les performances de nos algorithmes de couverture et de correction selon $3$ exp\'erimentations. 

% experimentation 1
La premi\`ere exp\'erimentation consiste \`a modifier les $k$ cases de la matrice d'adjacence du line-graphe d'un r\'eseau \'electrique. Ces cases modifi\'ees sont divis\'ees en deux sous-ensembles disjoints (cases {\em fausses n\'egatives} et cases {\em fausses positives}) selon une variable $p \in [0,1]$. Si $p = 0$ alors l'ensemble des cases modifi\'ees est compos\'e que de cases  {\em fausses positives} et si $p=1$ alors nous avons que des cases {\em fausses n\'egatives}. 
Le but est de borner le nombre de cases corrig\'ees par nos algorithmes.
Ainsi, nous avons d\'efini les distances de correction et de Hamming. 
La distance de correction est le nombre minimum de cases \`a modifier dans un graphe de $k$ cases \'erron\'ees pour en faire un line-graphe. 
Quant \`a la distance de Hamming, elle est la diff\'erence de cases entre les matrices de line-graphe propos\'e par nos algorithmes et le line-graphe du r\'eseau \'electrique.
Nous avons compar\'ee le nombre de cases corrig\'ees avec $5$ approches de correction qui sont : {\em al\'eatoire sans remise $(2c)$}, {\em degr\'e minimun sans remise $(2a)$}, {\em co\^ut minimum sans remise $(2b)$}, {\em degr\'e minimun avec remise $(1a)$}, {\em co\^ut minimum avec remise $(1b)$}. \`A chaque approche, nous avons $3$ co\^uts de modification  d'une case : {\em unitaire}, {\em ajout} et {\em suppression}.
Nous avons conclut que l'approche  {\em al\'eatoire sans remise $(2c)$} proposait des distances de correction minimales quelle que soit la repartition effectu\'ee $p$ et la fonction de co\^ut utilis\'ee. Ces distances constituent la borne sup\'erieure de la distance line quand le nombre $k$ de cases modifi\'ees est faible $k \le 5$. 
D'autre part, nous avons montr\'e que les distances de correction et de Hamming deviennent tr\`es corr\'el\'ees quand le nombre de cases modifi\'ees initiales est \'elev\'e $k > 10$. Dans ce cas o\`u $k \le 5$, le line-graphe propos\'e par l'algorithme de correction est celui de r\'eseau \'electrique. La distance de correction peut \^etre utilis\'ee comme une m\'etrique lorsque la topologie initiale du r\'eseau est inconnue.
\newline

% experimentation 2
La seconde exp\'erimentation consid\`ere que chaque case de la matrice du line-graphe est associ\'ee \`a une valeur de corr\'elation.  Les valeurs de corr\'elation sont g\'en\'er\'ees en tenant compte des distributions des valeurs de corr\'elation  du r\'eseau \'electrique d'un datacenter {\em Champlan}. Ces corr\'elations sont calcul\'ees avec la {\em distance de Pearson}. Les valeurs de corr\'elation dans la matrice forment la {\em matrice de corr\'elation}. Nous avons d\'efini un ensemble de seuil dans lequel chaque seuil est appliqu\'e \`a la matrice de corr\'elation pour en construire la matrice d'adjacence du graphe $G_s$. Le graphe $G_s$ contient des cases {\em fausses n\'egatives} et des cases {\em fausses positives}. 
Notre objectif est de minimiser le nombre de cases \'erronn\'ees dans le graphe $G_s$ apr\`es l'ex\'ecution de nos algorithmes et cela n\'ecessite la s\'election d'une valeur ad\'equate du seuil. 
Nous avons consid\'er\'e l'approche de correction  {\em al\'eatoire sans remise $(2c)$} et nous avons s\'electionn\'e quatre fonctions de co\^ut : {\em unitaire}, {\em ajout}, {\em suppression} et {\em normale}. Les fonctions de co\^ut sont fonction des cases modifi\'ees par l'algorithme de correction.
Nous avons d\'eduit que le bon seuil appartient \`a l'intervalle $s \in ]0.6,0.7]$ et la fonction {\em normale} donne de bons r\'esultats pour le calcul dans les distances de correction et de Hamming.
\newline

% experimentation 3
La derni\`ere exp\'erimentation se focalise sur les graphes dans lesquelles un sommet et son voisinage ne peuvent \^etre couverts par une ou deux cliques. Un exemple de ces graphes est la famille des graphes {\em grilles boucl\'ees}. Une grille boucl\'ee de  $k$  lignes et $k'$ colonnes est compos\'ee de $k \times k' +1$ cellules avec une cellule un graphe biparti $K_{2,2}$ non orient\'e.
Nous avons montr\'e que la correction de ces graphes peut se faire selon deux m\'ethodes ( modification  {\em ajout d'ar\^etes uniquement} et {\em suppression d'ar\^etes uniquement}). Les deux modifications admettent la m\^eme borne sup\'erieure de ses distances line. 
 Notre objectif est de v\'erifier que la convergence  des distances de correction  vers la borne sup\'erieure quelque soit la modification r\'ealis\'ee. 
 Les r\'esultats obtenus montrent que les distances de correction sont invariantes peu importe les modifications et que les  graphes corrig\'es n'ont pas de distances de corrections optimales. 
 \newline
% la convergence des distances line obtenues apr\`es l'algorithme de correction ({\em distances line calcul\'ees}) vers les distances th\'eoriques.
%Nous avons construit $49$ graphes cellules $G_k$ de $k$ lignes et $k$ colonnes avec $k \le 49$ et $k$ un nombre impair. Pour chaque graphe $G_k$, nous appliquons l'algorithme de correction avec les fonctions {\em suppression} et {\em ajout} puis nous comparons les distances line calcul\'ees avec celles th\'eoriques.
%Nous  avons conclut que les distances line calcul\'ees varient tr\`es faiblement entre les fonctions {\em ajout} et {\em suppression} et ces distances ne convergent pas vers les distances th\'eoriques parce que l'algorithme de correction ajoute des ar\^etes dans la fonction {\em suppression} et supprime des ar\^etes dans la fonction {\em ajout}.  
%Par ailleurs, les distances line th\'eoriques sont identiques quelques soient les fonctions {\em ajout} et {\em suppression} car les expressions litt\'erales de ces distances  ont le m\^eme d\^egre de polyn\^omes et les coefficents de ces polyn\^omes sont tr\`es proches.
%%La derni\`ere exp\'erimentation se focalise sur les graphes {\em iourtes}.
%%Un graphe iourte est un graphe dans lequel chaque sommet et son voisinage ne peuvent \^etre partitionn\'ees en deux cliques. Nous avons montr\'e que les distances line dans les graphes iourtes sont major\'ees. Notre objectif est de savoir si les distances line propos\'es par nos algorithmes convergent vers cette borne sup\'erieure.
%%Nous avons construit des graphes iourtes de profondeurs  diff\'erentes $k \in \{0,\times,50\}$
%%Les distances line dependent des fonctions de co\^ut {\em unitaire}, {\em ajout} et {\em suppression} et nous avons compar\'e ces fonctions de co\^ut en fonction de l\'ecart entre la distance line et la borne sup\'erieure.
%%Nous avons conclut que la fonction {\em suppression} propose de meilleurs distances line quand la profondeur augmente $k \ge 20$. Toutefois, aucune des fonctions ne converge vers la borne sup\'erieure. 
%\newline
%
%%Au terme de ces $3$ exp\'erimentations, nous pouvons conclure nos algorithmes ont un comportement optimal lorsque le mode de correction est {\em al\'eatoire sans remise}  et le seuil de corr\'elation est contenu dans l'intervalle $]0.6,0.7]$ peu importe la repartition des cases \'erronn\'ees et la fonction de co\^ut. Ces conditions garantissent que la matrice du graphe contienne peu de cases \'erronn\'ees et que ces cases seront corrig\'ees pendant l'algorithme de correction. Cependant pour la famille des graphes dont la couverture de corr\'elation est vide (cas des graphes {\em iourtes}),  la fonction {suppression} donne de meilleurs r\'esultats en termes de distances.

Au terme de ces $3$ exp\'erimentations, nous pouvons conclure nos algorithmes n'ont pas un comportement optimal lorsque le mode de correction est {\em al\'eatoire sans remise}  et le seuil de corr\'elation est contenu dans l'intervalle $]0.6,0.7]$ peu importe la repartition des cases \'erron\'ees et la fonction de co\^ut. 
Ces conditions garantissent que la distance de correction est minimale pour des graphes ayant peu d'erreurs et le line-graphe propos\'e par l'algorithme de correction diff\`ere de peu d'ar\^etes du line-graphe cible. 
 Cependant pour la famille des graphes dont aucun sommet ne peut \^etre couvert par une clique (cas des grilles {\em boucl\'ees}),  les distances line calcul\'ees ne convergent pas vers la borne sup\'erieure d\'efinie. Toutefois, le type de correction (modifications {\em ajout uniquement} et {\em suppression uniquement}) sur ces graphes n'influencent pas les valeurs de distances de correction. 


