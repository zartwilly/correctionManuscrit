
Les grilles boucl\'ees ont la particularit\'e d'avoir des ar\^etes qui ne peuvent \^etre partitionn\'ees en cliques. 
Nous avons d\'ecrit la construction de ces graphes et nous d\'efinissons deux m\'ethodes pour les corriger. La premi\`ere m\'ethode est la modification d'{\em ajout uniquement} qui consiste \`a ajouter uniquement des ar\^etes  et la seconde m\'ethode est la modification {\em suppression uniquement} qui supprime uniquement des ar\^etes des grilles boucl\'ees. 
Avec ces m\'ethodes, nous avons trouv\'e une borne sup\'erieure unique de la distance line des  grilles boucl\'ees et nous avons compar\'e cette borne sup\'erieure avec les distances de correction obtenues apr\`es l'algorithme de correction.
\newline
Nous remarquons que les distances de correction varient tr\`es peu entre les modifications {\em ajout uniquement} et {\em suppression uniquement}.  Les graphes corrig\'es n'ont pas de distances de corrections optimales parce que l'algorithme supprime des ar\^etes pendant la modification {\em ajout d'ar\^etes uniquement} et ajoute des ar\^etes pendant la modification {\em suppression uniquement}.  

%et ces distances ne convergent pas vers les distances th\'eoriques parce qu'il ajoute des ar\^etes dans l'op\'eration {\em suppression uniquement} et supprime des ar\^etes dans l'op\'eration {\em ajout uniquement}.  Par ailleurs, les distances line th\'eoriques sont identiques quelques soient les op\'erations car les expressions litt\'erales de ces distances  ont le m\^eme d\^egre de polyn\^omes et les coefficents de ces polyn\^omes sont tr\`es proches.