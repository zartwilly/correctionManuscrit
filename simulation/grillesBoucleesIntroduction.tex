Nous consid\'erons des graphes dans lequels le voisinage d'un sommet peut \^etre couvert par  une ou deux cliques. L'ex\'ecution de l'algorithme de couverture sur chacun de ces graphes fournit une couverture vide. Cette famille de graphes est d\'esign\'ee graphes {\em grilles boucl\'ees}. Apr\`es l'ex\'ecution de l'algorithme de couverture, tous les sommets de la {\em grille boucl\'ee} sont dans l'ensemble $\cal C$ des sommets \`a corriger.
\newline
Dans cette section, nous \'evaluons les performances de nos algorithmes, particuli\`erement l'algorithme de correction en effectuant  des op\'erations de {\em suppression } et d'{\em ajout} d'ar\^etes uniquement.
Nous d\'eterminons une borne sup\'erieure de la distance line de ces graphes.
Nous comparons cette borne avec les r\'esultats obtenus par l'algorithme de correction avec l'approche {\em al\'eatoire sans remise} (voir tableau \ref{tab:recapApprocheCorrection}).
\newline
 Nous d\'ebutons notre analyse par la d\'efinition et la construction d'une grille boucl\'ee. Ensuite, nous d\'ecrivons le protocole d'exp\'erimentation sur des grilles boucl\'ees d'ordres diff\'erents. Enfin nous interpr\'etons les r\'esultats obtenus pour chaque op\'eration.