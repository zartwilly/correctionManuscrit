Nous allons comparer cette borne sup\'erieure de l'\'equation \ref{borneSuperieureDL} avec les distances de correction obtenues par l'algorithme de correction.
\newline
Consid\'erons $G_{k,k'}$ une grille boucl\'ee dans laquelle le nombre de sommets par lignes est identique le nombre de sommets par colonnes ($k = k'$). Nous le notons $G_k$.
\newline
Nous construisons $48$ grilles boucl\'ees  contenant chacune $k \times k +1$ cellules, avec $k \in \{2,\cdots,98\}$ un nombre pair.
Dans chaque graphe $G_k$, nous ex\'ecutons   $50$ fois notre couple d'algorithmes avec chaque modification d'ar\^etes et la distance de correction obtenue est compar\'ee avec la borne sup\'erieure (\'equation \ref{borneSuperieureDL}).
\newline

Soient $\phi^{+}(u,v)$ le co\^ut de l'op\'eration {\em ajouter une ar\^ete} et 
$\phi^{-}(u,v)$ le poids de l'op\'eration {\em supprimer une ar\^ete} (voir section \ref{algorithmeCorrection}). 
\newline
La modification {\em ajout d'ar\^etes uniquement} est telle que 
\begin{itemize}
	\item L'ajout d'ar\^etes a un co\^ut  $\phi^{+}(u,v) = 1$,
	\item La suppression d'ar\^etes a un co\^ut  $\phi^{-}(u,v) = 10$.
\end{itemize}
Quant \`a la modification {\em suppression d'ar\^etes uniquement}, elle se d\'efinit comme suit :
\begin{itemize}
	\item L'ajout d'ar\^etes a un co\^ut  $\phi^{+}(u,v) = 10$.
	\item La suppression d'ar\^etes a un co\^ut  $\phi^{-}(u,v) = 1$.
\end{itemize}
Nous allons comparer l'\'evolution des distances de correction des $48$ graphes en fonction la borne sup\'erieure pour chaque modification r\'ealis\'ee.

%
%Consid\'erons $G_{k,k'}$ une grille boucl\'ee dans lequel le nombre de sommets par lignes est identique le nombre de sommets par colonnes ($k = k'$). Nous notons $G_k$ pour d\'esigner $G_{k,k'}$ et $G_k$ contient $k \times k +1$ cellules.
%Nous construisons $48$ grilles boucl\'ees dans lesquels chaque graphe contient $k \times k +1$ cellules $k \in \{2,\cdots,98\}$.
%Dans chaque graphe $G_k$, nous ex\'ecutons  notre couple d'algorithmes $50$ fois puis nous r\'ecup\'erons la distance de correction  minimale entre $G_k$ et les line-graphes $L(G_k)$ propos\'ees.
%Nous utilisons la   distance de correction  minimale parce qu'il est impossible de trouver la bonne permutation qui minimise cette distance \`a cause de la combinatoire factorielle de l'ensemble $\cal C$.
%\newline
%Soient $\phi^{+}(u,v)$ le poids de l'op\'eration {\em ajouter une ar\^ete} et $\phi^{-}(u,v)$ le poids de l'op\'eration {\em supprimer une ar\^ete}. 
%L'op\'eration {\em ajout uniquement} d'ar\^etes est telle que  $\phi^{+}(u,v) = 1$ pour l'ajout de l'ar\^ete $(u,v)$ et $\phi^{-}(u,v) = 10$ pour la suppression de l'ar\^ete $(u,v)$.
%De m\^eme,  L'op\'eration {\em suppression uniquement} d'ar\^etes est telle que  $\phi^{+}(u,v) = 10$ pour l'ajout de l'ar\^ete $(u,v)$ et $\phi^{-}(u,v) = 1$ pour la suppression de l'ar\^ete $(u,v)$.
%L'objectif de notre exp\'erimentation est de comparer l'\'evolution des distances de correction obtenues apr\`es la correction des grilles boucl\'ees par rapport aux bornes sup\'erieures des distances line th\'eoriques. 

%Nous d\'efinissons les fonctions de co\^ut {\em ajout} et {\em suppression} comme suit :
%\begin{enumerate}[label = (\alph*)]
%%\item {\em unitaire} : ajouter et supprimer une ar\^ete ont un poids $1$ ($\phi^{+} = \phi^{-} = 1$).
%\item {\em ajout} : nous donnons un poids minimal \`a l'op\'eration ``ajouter d'ar\^etes''. Ainsi nous attribuons un poids $\phi^{+} = 1$ pour l'ajout d'ar\^etes et un poids $\phi^{-} = 10$ pour la suppression d'ar\^etes.
%\item {\em suppression} : Ici nous faisons l'op\'eration inverse en donnant un poids maximal \`a l'op\'eration d'ajout d'ar\^etes. Ainsi, l'ajout d'une ar\^ete a un poids $\phi^{+} = 10$ alors que la suppression a un poids $\phi^{-} = 1$.
%\end{enumerate}
%L'objectif de notre exp\'erimentation est de comparer les distances lines th\'eoriques avec celles obtenus avec l'utilisation des fonctions de co\^uts mentionn\'ees ci-dessus.


%Nous allons comparer les diff\'erentes fonctions de co\^ut en utilisant les distances line calcul\'ees. Les courbes croissantes dans les figures \ref{priorAjout1Ajout10}, \ref{priorAjout1Supp10} et \ref{priorAjout1Supp1} s'explique par le fait que les distances line sont rang\'ees par ordre croissant.