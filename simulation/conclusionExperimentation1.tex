
%----- conclusion 
Les performances de notre couple d'algorithmes ont \'et\'e test\'ees sur des graphes non orient\'es sans circuits qui repr\'esentent la topologie de r\'eseaux \'electriques. Nous avons construit les line-graphes de ces graphes et avons modifi\'e $k$ cases dans les matrices des line-graphes. Les cases modifi\'ees sont reparties en deux ensembles (cases {\em fausses positives} et {\em fausses n\'egatives}) selon la variable $p \in [0,1]$.
L'analyse des performances compare les distances de correction et de Hamming en fonction du nombre de cases modifi\'ees selon $5$ approches de correction et $3$ fonctions de co\^ut (voir tableau \ref{tab:recapApprocheCorrection}). 
Nous concluons que l'approche {\em al\'eatoire sans remise} donne de meilleurs r\'esultats lorsque le nombre $k$ de cases modifi\'ees est faible ($k\le 5$). La distance line est alors major\'ee par la distance de correction.  Le probl\`eme {\em Proxi-Line} a une solution mais elle n'est pas optimale.
En revanche, au-d\'el\`a de $k > 5$, il est alors difficile de d\'eterminer une borne sup\'erieure \`a la distance line car l'algorithme de correction  modifie des cases n'\'etant pas contenues dans les $k$ cases modifi\'ees pr\'ealablement. 
Par ailleurs, nous avons montr\'e que la distance de correction peut \^etre utilis\'ee comme une m\'etrique pour connaitre le nombre de cases modifi\'ees dans la matrice d'un graphe \`a condition que cette distance soit sup\'erieure \`a $10$ ar\^etes. 
Enfin, les fonctions de co\^ut ont peu d'influences sur les distances de correction pendant la phase de correction.

%----- conclusion
