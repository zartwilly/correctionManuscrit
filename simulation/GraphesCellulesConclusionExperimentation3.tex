
Les graphes cellules ont la particularit\'e d'avoir des ar\^etes qui ne peuvent \^etre partitionn\'ees en cliques. Nous avons definis deux m\'ethodes pour corriger ces  graphes. La premi\`ere m\'ethode est d\'ecrite par la fonction {\em ajout} qui consiste \`a ajouter uniquement des ar\^etes  et la seconde m\'ethode est d\'ecrite par la fonction {\em suppression} qui supprime uniquement des ar\^etes des graphes cellules. Chaque m\'ethode fournit des line-graphes de distance line optimale et nous avons exprim\'e cette distance line en fonction l'ordre des graphes cellules. Nous les avons nomm\'ees {\em distances line th\'eoriques}.
\newline
Apr\`es l'ex\'ecution de notre couple d'algorithmes, nous remarquons que les distances line varient tr\`es faiblement entre les fonctions {\em ajout} et {\em suppression} et ces distances ne convergent pas vers les distances th\'eoriques parce qu'il ajoute des ar\^etes dans la fonction {\em suppression} et supprime des ar\^etes dans la fonction {\em ajout}.  
Par ailleurs, les distances line th\'eoriques sont identiques quelques soient les fonctions {\em ajout} et {\em suppression} car les expressions litt\'erales de ces distances  ont le m\^eme d\^egre de polyn\^omes et les coefficents de ces polyn\^omes sont tr\`es proches.