La topologie du r\'eseau \'electrique est repr\'esent\'ee par un graphe orient\'e sans circuit $G$.
Les c\^ables \'electriques sont unidirectionnels et les \'equipements sont toujours aliment\'es par une source. Ce qui implique que le courant se propage dans une direction et cette direction  indique l'orientation des arcs d'un {\em DAG} ({\em Directed Acyclic Graph}).
Nous allons d\'ecrire comment nous g\'en\'erons le graphe $G$.
\newline
Consid\'erons un graphe non orient\'e $G=(V, E)$ dans lequel  $V$ est l'ensemble de $n$ sommets, $E$ l'ensemble des $m$ ar\^etes et $\alpha$ son degr\'e moyen choisi. 
La probabilit\'e d'existence d'une ar\^ete entre deux sommets est de $\frac{\alpha}{n}$. 
Afin de g\'en\'erer un tel graphe apr\`es avoir choisir $n$ et $\alpha$, nous s\'electionnons  deux sommets $x$ et $y$ de $V$ et nous g\'en\'erons une valeur $p_{xy}$ qui suit  une loi de probabilit\'e uniforme.
Si $p_{xy}$ est sup\'erieure \`a la probabilit\'e d'existence d'une ar\^ete alors nous ajoutons l'ar\^ete $(x,y)$ \`a $E$.
%Si $p_{xy} \ge \frac{\alpha}{n} $ alors on ajoute une ar\^ete \`a $G$.
\newline
Si $G$ n'est pas connexe, nous choisissons al\'eatoirement un sommet dans chaque composante connexe et nous ajoutons une ar\^ete entre ces sommets.  Nous obtenons alors  $m$ ar\^etes.
\newline
Afin d'orienter $G$ comme un $DAG$, nous s\'electionnons de fa\c con al\'eatoire quatre sommets  de degr\'e minimum pour les d\'efinir comme les sources de notre tri topologique.
Nous effectuons ce tri avec un parcours en largeur {\em Breast First Search (BFS)} dans le graphe $G$. 
Chaque sommet $x$ obtient un ordre topologique $D_x$ et l'ar\^ete $e_{xy}$ devient 
soit l'arc $a_{xy}$ si $D_x < D_y$ 
soit l'arc $a_{yx}$ si $D_x > D_y$. 
%L'ensemble $E=\{e_{xy}, \forall a_{xy} \in A\}$ contient les arcs du graphe $G$.
Les arcs $a_{xy}$ forment l'ensemble $A$ des arcs de $G$. 
Nous en d\'eduisons que $G=(V, A)$ est orient\'e et son line-graphe $LG$ est construit \`a partir de la d\'efinition \ref{definitionLineGraphe}.
\newline
Nous notons $M_{G}$  la matrice d'adjacence de $G$ et $M_{LG}$ la matrice d'adjacence du line-graphe $LG$.
