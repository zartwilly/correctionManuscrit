Les graphes cellules sont des graphes dans lesquels chaque sommet est couvert par plus de $2$ cliques. Ce qui implique la couverture de corr\'elation est vide parce que  le couverture des ar\^etes de cette famille de graphes  est impossible. Tous les sommets de ces graphes sont alors contenus dans l'ensemble ${\cal C}$ des sommets \`a corriger.
\newline
Dans cette section, nous montrons que la correction de ces graphes peut se r\'ealiser en ajoutant ou en supprimant des ar\^etes uniquement.
Nous consid\'erons deux fonctions de co\^ut {\em ajout} et {\em suppression} qui correspondent respectivement \`a l'ajout et la suppression d'ar\^etes pendant l'algorithme de correction. Puis nous v\'erifions si la correction des graphes cellules convergent vers les bornes sup\'erieures des distances line de chaque fonction. 
Pour ce faire, nous corrigeons les sommets de  ${\cal C}$ avec le mode {\em al\'eatoire sans remise} en utilisant les fonctions {\em ajout} et {\em suppression}.
\newline
Nous d\'ebutons notre analyse par la d\'efinition et la construction d'un graphe cellule. Ensuite, nous d\'ecrirons le protocole d'exp\'erimentation sur des graphes cellules d'ordres diff\'erents. Enfin nous interpr\'etons les r\'esultats obtenus pour chaque fonction de co\^ut.