

Par ces mots s'ach\`eve la r\'edaction de ce manuscrit.
Je voudrais apporter ma reconnaissance \`a toutes ces personnes qui m'ont accompagn\'e durant cette p\'eriode en particulier \`a ma famille, aux coll\`egues de DCbrainSAS et aux membres de l'\'equipe DAVID.
\newline 

Un grand merci \`a ma famille pour leur immense soutien. Ils ont \'et\'e patients, compr\'ehensifs et encourageants. Je pense \`a mon p\`ere pour ses encouragements, ma m\`ere pour son reconfort, ma tante Felicit\'e pour ses conseils et son hospitalit\'e, mes fr\`eres et soeurs Grace, Prisca, Serge et Maruis. 
\newline
La rencontre de ces personnes a \'et\'e d\'ecisive dans mes choix de vie. 
Mario de son Petionville, merci pour tes conseils \'eclair\'es de la vie.
Dimitri l'electrochoc de mon quotidien, merci pour ta vision optimiste 
Alice merci pour les relectures, tu m'as \'et\'e d'une grande aide.
Cher Henri-joel, l'ami qui est d\'evenu un fr\`ere, merci d'avoir cru j'y suis arriv\'e. Je t'\'ecouterai plus souvent.   
Adrienne merci pour ses pri\`eres et C\'ecile pour son amour.
\newline

% dcbrain
De Earthgrid \`a DCbrain, l'aventure a d\'ebut\'e par un stage. De $3$ personnes au d\'ebut, elle est devenue une startup de dimension nationale sous la direction des bretons Arnaud et Fron\c cois. 
Arnaud, je retiens de toi ton esprit de management et aussi ta mani\`ere de simplifier un probl\`eme.
Francois, je retiens  de toi ton  coup de main sur les graphes et ses astuces de programmation.
A Maxime, merci pour son assistance et ses conseils en python pour l'impl\'ementation des algorithmes.
Et Damien pour son aide et ses explications sur les s\'eries temporelles.
\newline
%Le regret c'est d'avoir rat\'e le projet ACR.

%labo
Je voudrai particuli\`erement exprimer ma gratitude \`a  mon encadreur Dominique. Ses d\'ecisions, ses recommendations et sa disponibilit\'e sont le fruit de ce manuscrit. 
Vous n'etes pas un professeur pour rien.
\newline

% autres 
A TOUS CEUX QUE JE N'AI PAS CIT\'ES.
Le silence n'est pas synonyme d'oubli; mes pens\'es vont vers vous ;
\newline

L'enseignement que je retiens de ce travail : \newline
{\em Tout est possible \`a condition de faire le premier pas et de ne pas baisser les bras.}
