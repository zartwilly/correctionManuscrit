Les mod\`eles de classification de s\'eries temporelles bas\'es sur des caract\'eristiques
descriptives  supposent  d'extraire  un  ensemble  de  caract\`eres qu'on  esp\`ere  \^etre
repr\'esentatif de la forme g\'en\'erale d'une s\'erie temporelle. Le plus commun\'ement,
ces caract\`eres sont quantifi\'ees pour former des "sacs de mots" (BoW pour "Bag of Words'').
Dans la r\'ecup\'eration d'information, l'approche {\em BoW} d'estimation de la fr\'equence des mots en ignorant leur localisation est tr\`es commune. L'id\'ee est d'estimer la fr\'equence d'occurences des caract\`eres des s\'eries puis  d'utiliser ces fr\'equences comme des ``features'' pour faire de la  classification.
\newline
Les approches suivantes diff\`erent uniquement par les caract\'eristiques extraites.
En effet, l'approche {\em Bag Of Pattern (BOP)} \cite{lin2012rotation} convertit la s\'erie temporelle en une s\'erie discr\`ete gr\^ace \`a la m\'ethode {\em Symbolic Aggregate approXimation (SAX)} \cite{lin2007experiencing}. Il cr\'ee un ensemble de mots {\em SAX} pour chaque s\'erie par l'application d'une fen\^etre glissante, puis se sert de la fr\'equence des mots dans la s\'erie comme sa nouvelle caract\'eristique. 
{\em Baydoyan et al.} \cite{baydogan2013bag} d\'ecrit l'approche {\em bag-of-features} qui combine les caract\'eristiques de fr\'equences et d'intervalles. L'algorithme appel\'e {\em time series based on bag-of-features representation (TSBF)} implique la s\'eparation entre la cr\'eation de features et les \'etapes de classification. 
La cr\'eation de features implique la g\'en\'eration d'intervalles al\'eatoires et les features repr\'esentent, g\'en\'eralement, la moyenne, la variance et la pente sur un intervalle. 
Le d\'ebut et la fin d'un intervalle sont incluses dans les features. 