Dans cette partie, nous avons \'enum\'er\'e les diff\'erentes m\'ethodes (distances) que nous pouvons utiliser pour calculer la similarit\'e entre des s\'eries. Nous avons cat\'egoris\'e les distances en deux groupes : 
celles qui n'apportent aucune modification aux s\'eries et 
celles qui transforment les s\'eries avant de d\'ebuter l'analyse. 
La transformation des s\'eries se fait de deux mani\`eres. La premi\`ere consiste \`a diviser la s\'erie en s\'equences et \`a supposer que chaque s\'equence est ind\'ependante des autres. 
Quant \`a la seconde, elle consiste \`a remplacer la s\'erie par certaines caract\'eristiques descriptives telles que la moyenne, l'\'ecart-type, l'encodage de la s\'erie par des mots. 
Parmi celles qui ne modifient pas les s\'eries, nous distinguons certaines qui ont la propri\'et\'e de m\'etriques. 
%Elle signifie que la distance entre la serie A et B est la meme que la distance entre B et A. 
Cette propri\'et\'e est importante pour choisir la m\'ethode de calcul de la similarit\'e entre des s\'eries.   