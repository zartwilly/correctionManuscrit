Les mesures sur les arcs forment des s\'eries temporelles.
Certains arcs partagent des \'equipements que nous souhaitons d\'eterminer \`a partir des s\'eries temporelles. Nous supposons que les  s\'eries temporelles associ\'ees \`a ces arcs ont les m\^emes comportements au cours du temps. En d'autres termes, toute variation dans une s\'erie est visible dans une autre s\'erie. Nous disons que ces arcs sont {\em corr\'el\'es} et la valeur li\'ee \`a cette relation entre les arcs est d\'esign\'ee par {\em coefficient de similarit\'e}.  
\newline
Dans le chapitre pr\'ec\'edent, 
nous avons mod\'elis\'e les mesures sur les arcs par des s\'eries temporelles et la particularit\'e de ces s\'eries  est la pr\'esence de valeurs \'erron\'ees et manquantes. 
% quel est notre probleme
\newline
Le probl\`eme est de savoir s'il existe une m\'ethode de calcul du coefficient de similarit\'e qui tienne compte des erreurs dans les s\'eries temporelles et qui v\'erifie l'hypoth\`ese sous-jacente.
\newline
Pour ce faire, nous proc\'edons comme suit : la premi\`ere partie \'enonce  les analyses sur les s\'eries temporelles. La seconde partie pr\'esente les diff\'erentes m\'ethodes de calcul des coefficients de similarit\'e entre les s\'eries. Et enfin la derni\`ere partie s\'electionne la m\'ethode de calcul du {\em coefficient de similarit\'e} puis analyse les performances de cette m\'ethode sur les donn\'ees r\'eelles d'un {\em sous-r\'eseau du datacenter Champlan}.  