{\em Collective of Transform Ensemble (COTE)} se base sur deux id\'ees qui permettent la transformation des donn\'ees et am\'eliore la pr\'ecision des algorithmes de classification de s\'eries temporelles. La premi\`ere id\'ee est la transformation de nos donn\'ees dans un espace alternatif  o\`u les selections des caract\'eristiques des donn\'ees {\em features} est plus simple \`a detecter. La seconde id\'ee utilise les ensembles heterog\`enes qui produisent de meilleurs resultats qu'un re\'echantillonnage avec une apprentissage fiable.

provient de la demontration de {\em Keogh} \cite{} qui affirme que la pr\'ecision est atteinte par des ensembles de sch\'emas simples.  
{\em COTE} contient un ensemble de classifiers construit dans le domaine temporelle, frequentielle et dans la transformation de shapelets. Par exemple, COTE utilise des distances \'elastiques quand il op\`ere dans le domaine temporelle. 

it was demonstrated that with a single data representation, improved accuracy can be achieved through simple ensemble schemes. We combine these two principles to test the hypothesis that forming a collective of ensembles of classifiers on different data transformations improves the accuracy of time-series classification. The collective contains classifiers constructed in the time, frequency, change, and shapelet transformation domains.For the time domain, we use a set of elastic distance measures.