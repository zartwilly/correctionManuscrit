Dans le chapitre pr\'ec\'edent, nous avons d\'etermin\'e la matrice de corr\'elation $M_c$ du graphe $G$ d'un r\'eseau \'electrique. Cette matrice peut contenir des cases \'erron\'ees. Une case \'erron\'ee  $M_c[i,j]$ est un coefficient de corr\'elation proche de $1$ (resp. de $0$) entre les arcs $i$ et $j$ alors que ces arcs ne partagent aucune extr\'emit\'e (resp. ces arcs ont une extr\'emit\'e commune).
\newline
Nous consid\'erons une matrice $M$ de dimension identique \`a celle de $M_c$ telle que, pour toute valeur de seuil $s \in [0,1]$ et toute paire d'arcs $(i,j)$, $M[i,j] = 1$ si et seulement si $M_c[i,j] \ge s$. La matrice $M$ est la matrice d'adjacence d'un graphe non orient\'e $G_c$ dit {\em graphe de corr\'elation}. Cette matrice peut \'egalement contenir des cases \'erron\'ees.  
Une case \'erron\'ee $M[i,j] = 1$ d\'esigne la pr\'esence d'ar\^etes dans $G_c$ alors qu'il n'existe aucune ar\^ete entre les sommets $i$ et $j$ dans le line-grahe du graphe non-orient\'e sous-jacent au DAG $G$. 
De m\^eme, l'absence d'une ar\^ete entre $i$ et $j$ dans $G_c$ alors qu'elle est pr\'esente dans le line-grahe du graphe non-orient\'e sous-jacent \`a $G$ est aussi une case \'erron\'e  $M[i,j] = 0$.
\newline 
S'il n'existe aucune case \'erron\'ee dans la matrice $M$, alors $G_c$ est le line-graphe de graphe non-orient\'e sous-jacent au DAG $G$ et le line-graphe de $G$ est isomorphe \`a $G_c$.
Notre but est de d\'eduire le DAG $G$ \`a partir de $G_c$ en deux \'etapes :
\begin{itemize}
	\item D\'eterminer si $G_c$ est un line-graphe. Si c'est  le cas, d\'eduire le graphe dont $G_c$ est le line-graphe.
	\item Dans le cas o\`u $G_c$ n'est pas un line-graphe, proposer un algorithme qui modifie $G_c$ de telle sorte qu'il devient un line-graphe et ensuite d\'eduire le graphe dont le graphe $G_c$ modifi\'e est le line-graphe.
\end{itemize}
 Nous d\'esignons ce probl\`eme par {\em Proxi-Line}.
\newline
Nous allons, dans un premier temps, d\'ecrire les caract\'eristiques d'un line-graphe et le probl\`eme {\em Proxi-Line}. Ensuite nous pr\'esentons les algorithmes qui traitent ce probl\`eme et enfin nous expliquons la reconstruction de la topologie \`a partir du line-graphe d\'ecouvert par nos algorithmes. 



