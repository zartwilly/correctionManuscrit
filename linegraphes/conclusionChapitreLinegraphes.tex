%%% reformulation et correction 
Nous avons d\'ecrit la relation existante entre la matrice de corr\'elation et la notion de line-graphe. En effet, la matrice de corr\'elation appliqu\'ee \`a une valeur de seuil est la matrice d'adjacence d'un line-graphe si elle ne contient aucune erreur de corr\'elation.  Dans le cas o\`u elle poss\`ede des cases erron\'ees, il est impossible de d\'eterminer la {\em couverture de corr\'elation} du graphe de corr\'elation. Nous avons alors d\'efini le probl\`eme {\em Proxi-Line} dont l'objectif est de trouver le line-graphe  qui a le moins d'ar\^etes diff\'erentes avec le graphe de corr\'elation.  
\newline

Ensuite, nous avons pr\'esent\'e les propri\'et\'es d'un line-graphe et les travaux existants dans la d\'ecouverte de couverture en cliques sur des line-graphes. Nous avons retenu l'algorithme de Lehot \cite{decompositionEnCliques} comme la base de notre algorithme de couverture parce qu'il attribue  des \'etats \`a tous les sommets \`a chaque \'etape de l'algorithme. Cette op\'eration permet de connaitre les sommets non couverts des sommets d\'ej\`a couverts. 
\newline
Dans le but de r\'esoudre le probl\`eme {\em Proxi-Line}, nous proposons ainsi deux algorithmes. 
\newline
Le premier algorithme est {\em l'algorithme de couverture} et il couvre tous les sommets par une ou deux cliques \`a partir des \'etats des sommets. Nous avons distingu\'e trois types d'\'etats.
En effet, un sommet couvert par une clique et qui poss\`ede des ar\^etes incidentes est \`a l'\'etat  $2$. Un sommet couvert par une clique ou deux cliques et qui n'a aucune ar\^ete incidente est \`a l'\'etat $1$. Un sommet couvert par deux cliques ayant des ar\^etes incidentes ou un sommet ayant des ar\^etes incidentes qui ne forment pas une clique est \`a l'\'etat $-1$. 
L'ensemble des sommets $v$ \`a l'\'etat $-1$ est l'ensemble $\cal C$ de sommets \`a corriger. \`A la fin de l'algorithme, il retourne les cliques d\'ecouvertes (couverture de corr\'elation $\cal CC$). 
\newline
Le second algorithme propos\'e est {\em l'algorithme de correction}. Il se base sur l'ensemble $\cal C$ et les cliques d\'ecouvertes pendant l'algorithme de couverture. 
En effet, cet algorithme s\'electionne chaque sommet $u \in \cal C$ et le corrige en proc\'edant par une phase d'augmentation et de compression.
La phase d'augmentation d\'etermine les cliques de $\cal CC$ contenant $u$, les cliques de $\cal CC$  dont $u$ partage une ar\^ete avec un sommet de la clique (cliques voisines), les cliques contractables (cliques de $\cal CC$ dont l'intersection retourne le sommet $u$) et les cliques d\'ependantes (cliques contenant $u$ dans lesquelles un des sommets partagent une ar\^ete avec une clique de la couverture  de corr\'elation $\cal CC$). 
Puis elle effectue le produit cart\'esien de ces ensembles de cliques. Chaque \'el\'ement de ce produit est not\'e $\pi_1$ ou $\pi_2$.
Quant \`a la phase de compression, elle s\'electionne deux \'el\'ements du produit cart\'esien qu'elle note  $\pi_1$ et $\pi_2$, puis elle ajoute des ar\^etes de  $\pi_1$ et $\pi_2$ dans l'ensemble des ar\^etes initiales du graphe de correction pour en faire des cliques. Elle cr\'ee aussi l'ensemble $\pi_s$ des ar\^etes \`a supprimer afin que le sommet $u$ ne soit couvert que par deux cliques. 
Les cliques $\pi_1$ et $\pi_2$ sont ajout\'ees \`a la couverture de corr\'elation $\cal CC$. 
La complexit\'e des algorithmes de couverture et de correction est {\em pseudo-polynomiale} en fonction du degr\'e du graphe de corr\'elation. 
\newline 

Enfin, la derni\`ere section pr\'esente la construction de la topologie du graphe racine et son orientation. Pour la construction de la topologie, nous nous servons principalement de la couverture de corr\'elation. En effet, chaque clique de cette couverture de corr\'elation est un sommet dans le graphe racine. Si l'intersection de deux cliques est non vide alors il existe une ar\^ete entre les sommets correspondant \`a ces cliques dans le graphe racine. 
Concernant l'orientation des ar\^etes, nous nous servons de la fonction $Verif-correl$ (section \ref{VerifCorrel}) qui teste tous les bipartions possibles afin de trouvant les arcs incidents entrants et sortants d'un sommet du graphe racine. Nous avons montr\'e qu'il existe un ordre de sommets qui r\'eduit le nombre de bipartitions \`a tester. Toutefois cette solution n'est pas optimale. 
