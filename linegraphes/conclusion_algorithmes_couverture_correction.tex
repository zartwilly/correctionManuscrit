% --- a supprimer ce commenntaire cest une redite de l'algorithme de couverture
%Dans cette section, nous d\'ecrivons deux algorithmes. 
%Le premier algorithme est {\em l'algorithme de couverture} et il est bas\'e sur l'algorithme de Lehot \cite{decompositionEnCliques}. L'algorithme consid\`ere que chaque sommet peut avoir $4$ \'etats. Le premier \'etat $Cliq = 1$ est attribu\'e \`a des sommets couverts par une clique ou deux cliques et qui n'a aucune ar\^ete incidente. Le second \'etat $Cliq = 2$ est attribu\'e \`a des sommets couverts par une clique et qui poss\`ede des ar\^etes incidentes. Le troisi\`eme \'etat $Cliq = 3$ est attribu\'e \`a des sommets initialement n'ayant aucun \`etat ($Cliq=0$) et couvert par une clique. Et  le dernier \'etat $Cliq = -1$  est attribu\'e \`a des sommets couverts par deux cliques ayant des ar\^etes incidentes ou un sommet ayant des ar\^etes incidentes qui ne forme pas une clique. 
%\`A la fin de l'algorithme, les cliques d\'ecouvertes sont stock\'ees dans la line-couverture $\cal H$ et les sommets dont l'\'etat est $Cliq = -1$ sont contenus dans l'ensemble $\cal C$ des sommets \`a corriger.


Dans cette section, nous d\'ecrivons deux algorithmes. 
Le premier algorithme est {\em l'algorithme de couverture} qui attribue un \'etat \`a un sommet du graphe de corr\'elation en fonction des cliques qui le couvrent. L'ensemble de cliques est la couverture de corr\'elation ${\cal CC}$. La particularit\'e de la couverture de corr\'elation est que chaque sommet appartient \`a $1$ ou $2$ cliques. Lorsqu'un  sommet  n'est pas couvert par $1$ ou $2$ cliques, cela signifie que le graphe de corr\'elation n'est pas un line-graphe et ces sommets sont regroup\'es dans l'ensemble $\cal C$ de sommets \`a corriger.
\newline
%Le second algorithme present\'e est l'algorithme de correction dont l'objectif est de corriger les sommets de l'ensemble de sommets \`a corriger afin que $G_c$ devient le line-graphe le plus proche possible du DAG. La correction consiste \`a ajouter et supprimer des ar\`etes  incidentes \`a chaque sommet de sommets de tel sorte que la clique retenue soit de co\^ut minimum. Nous supposons que les co\^ut des op\'erations {\em ajouter une ar\^ete} et {\em supprimer une ar\^ete} sont connues.



Le second algorithme est {\em l'algorithme de correction}. Il consiste \`a ajouter ou \`a supprimer des ar\^etes au voisinage d'un sommet $u \in {\cal C}$ afin que la partition de ce sommet et son voisinage forme deux cliques. Pour assurer ces op\'erations d'ajout et de suppression, il utilise une phase d'augmentation et de compression. 
En effet, la phase d'augmentation d\'etermine les cliques de ${\cal CC}$ contenant $u$, les cliques de ${\cal CC}$  dont $u$ partage une ar\^ete avec un sommet de la clique (cliques voisines), les cliques contractables (cliques de ${\cal CC}$ dont l'intersection retourne le sommet $u$) et les cliques d\'ependantes (cliques contenant $u$ dans lesquelles un des sommets partagent une ar\^ete avec une clique de la couverture de corr\'elation ${\cal CC}$). 
Puis elle effectue le produit cart\'esien de ces ensembles de cliques. Chaque \'el\'ement de ce produit est not\'e $\pi_1$ ou $\pi_2$.
Quant \`a la phase de compression, elle s\'electionne deux \'el\'ements du produit cart\'esien qu'elle note  $\pi_1$ et $\pi_2$, puis elle ajoute des ar\^etes \`a  $\pi_1$ et $\pi_2$ dans l'ensemble des ar\^etes initiales du graphe de correction pour en faire des cliques. Elle cr\'ee aussi l'ensemble $\pi_s$ des ar\^etes \`a supprimer pour que le sommet $u$ ne soit couvert que par deux cliques. 
Les cliques $\pi_1$ et $\pi_2$ sont ajout\'ees \`a la couverture de corr\'elation ${\cal CC}$. 
\newline
Avec la d\'ecouverte de la couverture de corr\'elation ${\cal CC}$, nous allons construire le graphe racine de ce line-graphe dans la section suivante.


