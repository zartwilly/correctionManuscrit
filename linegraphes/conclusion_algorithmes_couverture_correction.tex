
Dans cette section, nous d\'ecrivons deux algorithmes. 
Le premier algorithme est {\em l'algorithme de couverture} qui attribue un \'etat \`a un sommet du graphe de corr\'elation en fonction des cliques qui le couvrent. L'ensemble de cliques est la couverture de corr\'elation ${\cal CC}$. La particularit\'e de la couverture de corr\'elation est que chaque sommet appartient \`a $1$ ou $2$ cliques. Lorsqu'un  sommet  n'est pas couvert par $1$ ou $2$ cliques, cela signifie que le graphe de corr\'elation n'est pas un line-graphe et ces sommets sont regroup\'es dans l'ensemble $\cal C$ de sommets \`a corriger.
\newline

Le second algorithme est {\em l'algorithme de correction}. Il consiste \`a ajouter ou \`a supprimer des ar\^etes au voisinage d'un sommet $u \in {\cal C}$ afin que la partition de ce sommet et son voisinage forment deux cliques. Pour assurer ces op\'erations d'ajout et de suppression, il utilise une phase d'augmentation et de compression. 
En effet, la phase d'augmentation d\'etermine les cliques de ${\cal CC}$ contenant $u$, les cliques de ${\cal CC}$  dont $u$ partage une ar\^ete avec un sommet de la clique (cliques voisines), les cliques contractables (cliques de ${\cal CC}$ dont l'intersection retourne le sommet $u$) et les cliques d\'ependantes (cliques contenant $u$ dans lesquelles un des sommets partagent une ar\^ete avec une clique de la couverture de corr\'elation ${\cal CC}$). 
Puis elle effectue le produit cart\'esien de ces ensembles de cliques. Chaque \'el\'ement de ce produit est not\'e $\pi_1$ ou $\pi_2$.
Quant \`a la phase de compression, elle s\'electionne deux \'el\'ements du produit cart\'esien qu'elle note  $\pi_1$ et $\pi_2$, puis elle ajoute des ar\^etes \`a  $\pi_1$ et $\pi_2$ dans l'ensemble des ar\^etes initiales du graphe de correction pour en faire des cliques. Elle cr\'ee aussi l'ensemble $\pi_s$ des ar\^etes \`a supprimer pour que le sommet $u$ ne soit couvert que par deux cliques. 
Les cliques $\pi_1$ et $\pi_2$ sont ajout\'ees \`a la couverture de corr\'elation ${\cal CC}$. 
\newline
Avec la d\'ecouverte de la couverture de corr\'elation ${\cal CC}$, nous allons construire le graphe racine de ce line-graphe dans la section suivante.


