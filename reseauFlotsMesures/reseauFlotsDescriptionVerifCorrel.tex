\label{VerifCorrel}
La fonction $Verif-correl$ d\'etermine le sous-ensemble d'arcs entrants et sortants d'un sommet du r\'eseau \'electrique en se basant sur les mesures physiques et les lois de conservation $R$ d\'efinies dans le paragraphe \ref{reglesLocales}. 
\newline
Soit $S \subset A$ l'ensemble  fini d'arcs incidents \`a un sommet  $v \in V$ et 
$x \in GP$ une grandeur physique 
telle que chaque arc $a \in S$ a un flot $gp_{a}^{x}$.
Nous partitionnons $S$ en deux sous-ensembles $S_1$ et $S_2$ tels que $S_1 \cap S_2  = \emptyset$.
\newline
La fonction $Verif-correl$ est bool\'eenne, prend en param\`etres $S_1$, $S_2$ et une grandeur $x$. 
Elle retourne $1$ si :
\begin{itemize}
	\item $S_1$ est l'ensemble des arcs entrants du sommet $v$.
	\item $S_2$ est l'ensemble des arcs sortants du sommet $v$.
\end{itemize} 
Si les lois $R$ ne sont pas v\'erifi\'ees alors $Verif-correl$ retourne $0$.
$$
Verif-correl(S_1, S_2, x) = 1 \Leftrightarrow  \sum_{ a_i \in S_1} gp_{a_i}^{x}(t) - \sum_{a_ j\in S_2} gp_{a_j}^{x}(t)  \le \epsilon 
$$
En effet, consid\'erons $t$ un instant de temps et 
		$diff(t) =  \sum_{ a_i \in S_1} gp_{a_i}^{x}(t) - \sum_{a_ j\in S_2} gp_{a_j}^{x}(t)$ la diff\'erence entre les flots entrants et sortants \`a un instant $t$.
La diff\'erence $diff(t)$ doit toujours \^etre inf\'erieure aux pertes par effets joules $\epsilon$ quelle que soit l'instant $t$. 
\newline
Nous d\'ecidons que la fonction $Verif-Correl$ retourne $0$ si plus de $10\%$ des diff\'erences $diff(t)$ sont sup\'erieures \`a  $\epsilon$.
\newline


Nous consid\'erons, dans la suite du rapport, que la {\em d\'ecision de $Verif-correl$ est toujours exacte}. Cela signifie que 
si $Verif-correl(S1, S2, x) = 0$ alors les arcs de $S$ ne partagent pas un sommet de $V$.
%Cependant, dans la pratique, nous utilisons  rarement cette fonction parce que nous devons tester dans l'ordre de $2^{|S|}$ bipartitions dans le pire des cas pour obtenir la bonne partition et cela est impossible pour $S$ tr\`es grand. 
Cependant, dans la pratique, nous n'utilisons pas cette fonction pour d\'eterminer la topologie car pour un sommet $v$ de la topologie ayant un ensemble d'arcs $S_v$, nous devons  tester dans l'ordre de $2^{|S_{v}|}$ bipartitions dans le pire des cas pour obtenir la bonne bipartition et cela est impossible pour $S_{v}$ tr\`es grand. 
%La valeur $\epsilon$ dans le r\'eseau est connue. 
%Dans le cas o\`u cette valeur $\epsilon$ est inconnue, quel est la valeur minimum de ces pertes  \`a partir de laquelle une partition de $S$  identifie un sommet dans $V$?