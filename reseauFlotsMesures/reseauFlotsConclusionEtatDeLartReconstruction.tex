\newline
{\bf Conclusion} :
l'\'etude bibliographique effectu\'ee sur la d\'ecouverte de topologie montre que la recherche sur ce sujet  est tr\`es active dans le domaine des r\'eseaux informatiques, pr\'ecisement dans la d\'etection de n\oe uds/liens congestionn\'es. 
Toutefois, dans le domaine \'energ\'etique, les rares travaux r\'ealis\'es sur ce sujet mettent l'accent sur la d\'ecouverte de topologie par la reconstruction des sous-graphes en supposant que les n\oe uds du r\'eseau sont connus, 
que certains liens sont absents, 
que les mesures sont influenc\'ees par les incidents 
et que des erreurs peuvent \^etre pr\'esentes sur nos donn\'ees. 
Ces travaux s'av\`erent moins pertinents pour notre probl\'ematique parce  que les r\'esolutions propos\'ees s'appuyent sur des hypoth\`eses qui sont diff\'erentes des n\^otres. 
En effet, nous connaissons que les liens et les mesures qui circulent sur ces liens. 
Mais nous ignorons exactement les extr\'emit\'es de ces liens.
Par ailleurs, ces mesures suivent  des lois physiques qui impliquent la propagation d'\'ev\`enements dans ces r\'eseaux.
Notre probl\'ematique est alors de d\'eterminer les extr\'emit\'es partag\'ees entre les liens  gr\^ace aux lois et aux mesures physiques et aussi \`a la th\'eorie des graphes.


