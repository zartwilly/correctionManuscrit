
Soit $G=(V,A,CAP)$ le graphe orient\'e sans circuit mod\'elisant le r\'eseau \'electrique. 
Chaque sommet de $G$ est soit une source, soit  un tableau ou soit un serveur. 
L'ensemble des sommets $V$ est compos\'e des \'equipements sources $V_S$, interm\'ediaires ou passifs $V_I$ et  charges ou serveurs $V_C$. Il est une union disjointe, deux \`a deux, des partitions $V_C$, $V_I$ et $V_S$ de $V$ de cardinalit\'e $n$ dans laquelle :
\begin{itemize}
	\item Les sommets $V_S$ sont des sommets de degr\'e entrant nul $d^{-} = 0$.
	\item Les sommets $V_I$ sont des sommets de degr\'es entrant et sortant non nuls $d^{-} \ne 0, d^{+} \ne 0$.  
	\item Les sommets $V_C$ sont des sommets de degr\'e sortant nul  $d^{+} = 0$.
\end{itemize}
$$ 
V = V_S \cup V_I \cup V_C ~ et ~ 
V_S \cap V_I =  \emptyset ~ et ~ 
V_S \cap V_C =  \emptyset ~ et ~ 
V_C \cap V_I =  \emptyset
$$
L'ensemble des arcs $A$ mod\'elise $m$ c\^ables \'electriques.
Chaque arc a un flot, une capacit\'e et une r\'esistance consid\'er\'ee constante.
La capacit\'e de chaque arc est fonction de la grandeur associ\'ee \`a cet arc.
\newline
Par convention avec les \'equipes m\'etiers du r\'eseau, nous avons d\'ecid\'e que les arcs incidents entrants dans chaque sommet du graphe portent les mesures des grandeurs physiques.
