
Soit $G=(V,A,CAP)$ le graphe orient\'e sans circuit mod\'elisant le r\'eseau \'electrique. 
Chaque sommet de $G$ est soit une source, soit  un tableau ou soit un serveur. 
L'ensemble des sommets $V$ est compos\'e des \'equipements sources $V_S$, interm\'ediaires ou passifs $V_I$ et  charges ou serveurs $V_C$. Il est une union disjointe deux \`a deux  de $V_C$, $V_I$ et $V_S$ de cardinalit\'e $n$ dans laquelle :
\begin{itemize}
	\item Les sommets $V_S$ sont des sommets de degr\'e entrant nul $d^{-} = 0$.
	\item Les sommets $V_I$ sont des sommets de degr\'es entrant et sortant non nuls $d^{-} \ne 0, d^{+} \ne 0$.  
	\item Les sommets $V_C$ sont des sommets de degr\'e sortant nul  $d^{+} = 0$.
\end{itemize}
$$ 
V = V_S \cup V_I \cup V_C ~ et ~ 
V_S \cap V_I =  \emptyset ~ et ~ 
V_S \cap V_C =  \emptyset ~ et ~ 
V_C \cap V_I =  \emptyset
$$
L'ensemble des arcs $A$ mod\'elise $m$ c\^ables \'electriques.
Chaque arc a un flot, une capacit\'e et une r\'esistance consid\'er\'ee constante.
La capacit\'e de chaque arc est fonction de la grandeur associ\'ee \`a cet arc.
\newline
Par convention,  les arcs incidents entrants dans chaque sommet du graphe portent les mesures des grandeurs physiques.

%Consid\'erons $a_i^{-} \in V$ et $a_i^{+} \in V$ l'extr\'emit\'e initiale et l'extr\'emit\'e finale de l'arc $a_i \in E$ avec $1\le i \le n$.  
%Les sommets de l'arc  $a_i$ appartiennent \`a des ensembles differents tel que 
%\begin{itemize}
%\item si $a_i^{-} \in V_S$ alors $a_i^{+} \in V_I$.
%\item si $a_i^{-} \in V_I$ alors $a_i^{+} \in V_C$.
%\end{itemize}
%La capacit\'e de chaque arc est fonction de la grandeur associ\'e \`a cet arc.
%\newline
%Par convention,  les arcs incidents entrants dans chaque sommet du graphe portent les mesures physiques