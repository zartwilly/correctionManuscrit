%\newline

{\bf Conclusion} : 
le r\'eseau \'electrique d'un datacenter comprend des sources, des tableaux, des baies de serveurs et  ils sont tous reli\'es  par des c\^ables \'electriques. Les \'equipements de ce r\'eseau ne s'alimentent pas soi-m\^eme et les \'equipements de m\^eme nature ne poss\`edent pas de c\^ables entre eux. Les c\^ables sont undirectionnels et l'\'electricit\'e a toujours le m\^eme sens : de la source aux baies.
Nous en concluons que la topologie du datacenter est un {\em graphe orient\'e sans circuit} induit par le sens de la circulation du courant \'electrique sur ces liens. Ainsi ce graphe est la topologie du r\'eseau \'electrique. 
%et le fait que l'\'electricit\'e est toujours le m\^eme sens implique  que le r\'eseau \'electrique peut etre mod\'elis\'e par un graphe de flots avec la valeur de l'\'electricit\'e entre deux \'equipements comme le flot. 