L'\'electricit\'e, achemin\'ee par le gestionnaire de r\'eseau, arrive aux transformateurs basse tension qui g\'en\'eralement fonctionnent en mode triphas\'e.
Ces transformateurs vont convertir la puissance {\em HTA} re\c cue ($20$KVA chez Enedis) en une puissance {\em BT} ($400$KVA) et cette puissance est envoy\'ee sur chaque phase.
Une phase est un canal de transport de courant et 
le courant d'une phase est exprim\'e en fonction du sinus et d'un d\'ecalage de $2\pi/3$ par rapport au courant d'une autre phase.
Chaque phase transporte cette \'energie aux divers tableaux. 
Chaque tableau peut \^etre rattach\'e \`a deux phases pour \'eviter les micro-coupures d'\'electricit\'e. Pendant ces micro-coupures, les accumulateurs et les groupes \'electrog\`enes prennent le relai dans le but d'\'eviter une interruption de service.
Les tableaux sont rattach\'es aux phases et aux baies par des c\^ables \'electriques. 
Chaque \'equipement mesure la quantit\'e d'\'electricit\'e qui le traverse.
Les c\^ables sont unidirectionnels. 
Aucun \'equipement ne s'alimente lui-m\^eme et les \'equipements de m\^eme nature ne sont pas rattach\'es entre eux. Par exemple, il n'existe aucun c\^able entre des tableaux et aucune baie n'alimente une autre baie.
L'\'electricit\'e suit un sens : des sources vers les baies. 
Chaque \'equipement mesure la quantit\'e d'\'electricit\'e qui le traverse. 
 Par convention, ces mesures sont port\'ees par les c\^ables incidents entrants dans chaque \'equipement.
\newline
Notre r\'eseau \'electrique se  mod\'elise avec un r\'eseau de flots dont le graphe est un graphe orient\'e sans circuit {\em Directed Acyclic Graph (DAG)} dans lequel
chaque sommet repr\'esente un \'equipement,
un arc repr\'esente des c\^ables \'electriques et 
 qu'aucun \'equipement ne s'alimente soi-m\^eme (absence de circuits dans le r\'eseau). 