 \label{reglesLocales}
Un flot  $gp_{a}^{x}[t]$ est admissible s'il respecte, pour chaque arc $a \in A$ travers\'e, la contrainte ci-dessous:
\begin{equation}
	0 \le  gp_{a}^{x}[t] \le Cap_{a}[x]
\end{equation}	
avec $Cap_{a}$ la capacit\'e de l'arc $a$ pour la grandeur $x \in GP^{a}$.
\newline
Un flot est une fonction qui prend en entr\'ees 
un arc $a$, 
une grandeur $x \in GP^{a}$, 
un vecteur $gp_{a}^{x}$ associ\'e \`a la grandeur $x$ de l'arc $a$, 
un facteur de puissance $cos \phi$ ou $FP$ associ\'e  \`a l'arc $a$
et retourne un vecteur d\'efini comme suit :
\begin{equation}
	flo(a,x,gp_{a}^{x}) = f(a, x, r, cos \phi, gp_{a}^{x}) =
	\begin{cases}
		 \frac{  gp_{a}^{x} }{r \times cos \phi}, x \in gp_{ddp} \\
		 gp_{a}^{x} , x \in gp_{cal}
	\end{cases}
\end{equation}
avec $r$ la r\'esistance du c\^able, FP ou cos $\phi$ le facteur de puissance.
\newline
Une valeur $flo_{t}(a,x,gp_{a}^{x})$ de $flo(a,x,gp_{a}^{x})$ s'obtient \`a un indice $t < T^{a, x}$ donn\'e et se d\'efinit comme suit 
\begin{equation}
	flo_t(a,x,gp_{a}^{x}) = f(a, x, r, cos \phi, gp_{a}^{x}) =
	\begin{cases}
		 \frac{  gp_{a}^{x}(t) }{r \times cos \phi}, x \in gp_{ddp} \\
		 gp_{a}^{x}(t) , x \in gp_{cal}
	\end{cases}
\end{equation}
Soient $a \in A$ un arc et $x \in GP^{a}$ une grandeur li\'ee \`a l'arc $a$.
L'ensemble des arcs incidents \`a $a$ ayant la m\^eme extr\'emit\'e initiale que $a$ est not\'e  $succ(a)$ et 
 l'ensemble des arcs incidents \`a $a$ ayant la m\^eme extr\'emit\'e finale que $a$ est not\'e $pred(a)$.
 Tous les \'el\'ements de $succ(a)$ et $pred(a)$ ont les m\^emes grandeurs physiques. 
 \newline
La fonction $flo$ doit respecter la contrainte de la loi de conservation $R$ \cite{loiDeConservation}. La loi de  conservation $R$ ne s'applique qu'avec les grandeurs \`a effet calorique $gp_{cal} \in GP$ et se d\'efinit  comme suit :
\begin{enumerate}
		\item[] 
			\begin{equation}
				\sum_{a_{j} \in pred(a)} flo_t(a_{j},x,gp_{a_{j}}^{x}) = \sum_{a_{k} \in succ(a)} flo_t(a_{k},x,gp_{a_{k}}^{x}) + \epsilon 
			\end{equation}
		avec $\epsilon$ les pertes par Effet Joule. Cette \'equation est la loi de conservation ou de Kirchhoff.

\end{enumerate}

