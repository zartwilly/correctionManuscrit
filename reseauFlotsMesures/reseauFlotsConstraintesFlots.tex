 \label{reglesLocales}
Un flot  $gp_{a}^{x}[t]$ est admissible s'il respecte, pour chaque arc $a \in A$ travers\'e, la contrainte ci-dessous:
\begin{equation}
	0 \le  gp_{a}^{x}[t] \le Cap_{a}[x]
\end{equation}	
avec $Cap_{a}$ la capacit\'e de l'arc $a$ pour la grandeur $x \in GP^{a}$.
\newline
Un flot est une fonction qui prend en entr\'ees 
un arc $a$, 
une grandeur $x \in GP^{a}$, 
un vecteur $gp_{a}^{x}$ associ\'e \`a la grandeur $x$ de l'arc $a$, 
un facteur de puissance $cos \phi$ ou $FP$ associ\'e  \`a l'arc $a$
et retourne un vecteur d\'efini comme suit :
\begin{equation}
	flo(a,x,gp_{a}^{x}) = f(a, x, r, cos \phi, gp_{a}^{x}) =
	\begin{cases}
		 \frac{  gp_{a}^{x} }{r \times cos \phi}, x \in gp_{ddp} \\
		 gp_{a}^{x} , x \in gp_{cal}
	\end{cases}
\end{equation}
avec $r$ la r\'esistance du c\^able, FP ou cos $\phi$ le facteur de puissance.
\newline
Une valeur $flo_{t}(a,x,gp_{a}^{x})$ de $flo(a,x,gp_{a}^{x})$ s'obtient \`a un indice $t < T^{a, x}$ donn\'e et se d\'efinit comme suit : 
\begin{equation}
	flo_t(a,x,gp_{a}^{x}) = f(a, x, r, cos \phi, gp_{a}^{x}) =
	\begin{cases}
		 \frac{  gp_{a}^{x}(t) }{r \times cos \phi}, x \in gp_{ddp} \\
		 gp_{a}^{x}(t) , x \in gp_{cal}
	\end{cases}
\end{equation}
Soient $a \in A$ un arc et $x \in GP^{a}$ une grandeur li\'ee \`a l'arc $a$.
L'ensemble des arcs incidents \`a $a$ ayant la m\^eme extr\'emit\'e initiale que $a$ est not\'e  $succ(a)$ et 
 l'ensemble des arcs incidents \`a $a$ ayant la m\^eme extr\'emit\'e finale que $a$ est not\'e $pred(a)$.
 Tous les \'el\'ements de $succ(a)$ et $pred(a)$ ont les m\^emes grandeurs physiques. 
 \newline
La fonction $flo$ doit respecter la contrainte de la loi de conservation $R$ \cite{loiDeConservation}. La loi de  conservation $R$ ne s'applique qu'avec les grandeurs \`a effet calorique $gp_{cal} \in GP$ et se d\'efinit  comme suit :
\begin{enumerate}
		\item 
			\begin{equation}
				\sum_{a_{j} \in pred(a)} flo_t(a_{j},x,gp_{a_{j}}^{x}) = \sum_{a_{k} \in succ(a)} flo_t(a_{k},x,gp_{a_{k}}^{x}) + \epsilon 
			\end{equation}
		avec $\epsilon$ les pertes par effets joules. Cette \'equation est la loi de conservation ou de Kirchhoff

%		\item L'extr\'emit\'e finale de $a$ est un puits ou noeud consommateur
%			\begin{equation}
%				si~ succ(a) = \emptyset ~ alors ~ \sum_{a_{j} \in pred(a)} flo_t(a,x,gp_{a}^{x}) \ge 0 
%			\end{equation}	
%		\item L'extr\'emit\'e initiale de $a$ est une source ou noeud source
%			\begin{equation}
%				si ~ pred(a) = \emptyset ~ alors ~ \sum_{a_{j} \in succ(a)} flo_t(a,x, gp_{a}^{x}) \ge 0 
%			\end{equation}	
\end{enumerate}


%Soient 2 arcs $a_{i}, a_{j}$ ayant une extr\'emit\'e en commun, et une grandeur physique $gp_{a} \in GP$.\\
%Un flot $flo$ est une fonction qui prend en entr\'ee un arc et une grandeur physique et qui fournit une valeur r\'eelle en sortie. 
%Le flot d\'epend des caract\'eristiques  de l'arc (r\'esistance, cos $\phi$) et se d\'efinit ainsi:
%\begin{equation}
%	flo(a,gp_{a}^{x}) = f(a, r, cos \phi, gp_{a}^{x}) =
%	\begin{cases}
%		 \frac{  gp_{a}^{x} }{r \times cos \phi}, x \in gp_{ddp} \\
%		 gp_{a}^{x} , x \in gp_{cal}
%	\end{cases}
%\end{equation}
%avec $x \in GP^{a}$, $r$ la r\'esistance du c\^able, FP ou cos $\phi$ le facteur de puissance. \\
%Le vecteur de flot $flo(a,gp_{a}^{x})$ est de norme $T^{a,x}$.\\
%Une valeur de $flo(a,gp_{a}^{x})$ s'obtient \`a un indice $t < T^{a, x}$ donn\'e c'est-\`a-dire $flo_{t}(a,gp_{a}^{x})$
%\begin{equation}
%	flo_{t}(a, gp_{a}^{x}) = f[a, r, cos \phi, gp_{a}^{x}, t] =
%	\begin{cases}
%		 \frac{  gp_{a}^{x}[t] }{r \times cos \phi}, gp \in \{U, U_{12}, U_{23}, U_{13}  \} \\
%		 gp_{a}^{x}[t] , gp \in \{P, Q, I, I_{1}, I_{2}, I_{3}  \}
%	\end{cases}
%\end{equation}
%La loi de conservation aussi nomm\'ee  \textbf{r\`egles locales $R$} ne s'applique qu'avec les grandeurs \`a effet calorique $gp_{cal} \in GP$ et se d\'efinisse  comme suit:
%	\begin{enumerate}
%		\item 
%			\begin{equation}
%				\forall a_{i} \in A_{gp},  \sum_{a_{j} \in P(a_{i})} flo(a_{j},gp_{a_{j}}^{x}) = \sum_{a_{k} \in S( a_{i} )} flo(a_{k},gp_{a_{k}}^{x}) + \epsilon 
%			\end{equation}
%		avec $\epsilon$ les pertes par effets joules. Cette \'equation est la loi de conservation ou de Kirchhoff
%
%		\item l'extr\'emit\'e finale de $a_{i}$ est un puits ou noeud consommateur
%			\begin{equation}
%				si \hspace{0,1cm} S( a_{i} ) = \emptyset \hspace{0,1 cm} alors \hspace{0,1cm} \forall a_{i} \in A_{gp}, \sum_{a_{j} \in S(a_{i})} flo(a_{j},gp_{a_{j}}^{x}) \ge 0 
%			\end{equation}	
%		\item l'extr\'emit\'e initiale de $a_{i}$ est une source ou noeud source
%			\begin{equation}
%				si \hspace{0,1cm} P( a_{i} ) = \emptyset \hspace{0,1cm} alors  \hspace{0,1cm} \sum_{a_{j} \in P(a_{i})} flo(a_{j}, gp_{a_{j}}^{x}) \ge 0 
%			\end{equation}	
%	\end{enumerate}
%avec $ S(a_{i}) = \{a_{j}: a_{i}^{-} = a_{j}^{-}   \}$ l'ensemble des arcs ayant la m\^eme extr\'emit\'e initiale que $a_{i}$ et 
% $ P(a_{i}) = \{a_{j}: a_{i}^{+} = a_{j}^{+}   \}$  l'ensemble des arcs ayant la m\^eme extr\'emit\'e  finale que $a_{i}$.