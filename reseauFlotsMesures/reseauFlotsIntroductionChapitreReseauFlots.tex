Les r\'eseaux \'energ\'etiques ont pour r\^ole de fournir une \'energie \`a des entit\'es consommatrices sans interruption de service. Ces r\'eseaux fonctionnent en courant continu et en courant alternatif.
\newline
Notre \'etude est limit\'ee au courant alternatif parce que la demande d'\'electricit\'e des \'equipements varie constamment et la quantit\'e d'\'electricit\'e est connue pour chaque \'equipement de ce r\'eseau. 
Notre objectif est de regrouper les \'equipements qui ont la m\^eme source d'alimentation.
Ces sources d'alimentation subissent r\'eguli\`erement des maintenances et le sch\'ema \'electrique n'est pas mis \`a jour, ce qui entraine des probl\`emes dans le syst\`eme de surveillance de ce r\'eseau et aussi dans la planification de nouvelles maintenances.
\newline
Dans ce r\'eseau \'electrique, nous connaissons les liens et les mesures pr\'esentes sur ces liens mais nous ignorons les extr\'emit\'es de ces liens. Notre probl\`eme est d'identifier ces extr\'emit\'es \`a partir des mesures \'electriques. Nous faisons de la {\em d\'ecouverte de topologie}.  
Le mod\`ele propos\'e a \'et\'e \'etabli notamment \`a partir de la description du r\'eseau \'electrique d'un data center d'un op\'erateur t\'el\'ephonique appel\'e {\em Champlan}.
 \newline
Notre chapitre comprend quatre parties. La premi\`ere partie montre la relation entre un r\'eseau \'electrique et un graphe de flots. La seconde partie mod\'elise le r\'eseau de flots et la troisi\`eme partie pr\'esente les travaux existants sur la d\'ecouverte de topologie. 
La derni\`ere partie pr\'esente l'approche retenue pour r\'esoudre notre probl\`eme.