Dans ce chapitre, nous avons montr\'e que 
le r\'eseau \'electrique se mod\'elise par un graphe de flots dont les sommets sont des \'equipements, les arcs sont les c\^ables \'electriques unidirectionnels et les flots par les mesures des \'equipements. Par convention, ces mesures sont port\'ees par les arcs incidents entrants dans chaque sommet du graphe. 
Chaque mesure est associ\'ee \`a une grandeur physique et \`a chaque grandeur est d\'efinie une capacit\'e sur l'arc. Nous avons red\'efini la loi de conservation $R$ en fonction de nos mesures,  de nos grandeurs et nos capacit\'es.  
\newline
%Dans ce graphe, les arcs et leurs flots sont connues mais les sommets sont inconnues. Notre probl\`eme est de d\'eterminer les sommets communs aux arcs.
Dans ce graphe, nous connaissons les liens et les mesures qui circulent sur ces liens. 
Mais nous ignorons exactement les extr\'emit\'es de ces liens.
Par ailleurs, ces mesures suivent  des lois physiques qui impliquent la propagation d'\'ev\`enements dans ces r\'eseaux.
Notre probl\`eme est alors de d\'eterminer les extr\'emit\'es partag\'ees entre les liens  gr\^ace aux lois $R$, aux mesures physiques et aussi \`a la th\'eorie des graphes.
\newline
Une \'etude bibliographique a \'et\'e r\'ealis\'ee sur la d\'ecouverte de topologie. Les travaux concernent principalement la reconstruction de topologie dans le domaine informatique \`a partir de sondes et de protocoles de r\'eseau. Ces travaux s'accentuent sur la supervision de r\'eseau, l'administration du r\'eseau et aussi sur la recherche des caract\'eristiques des \'el\'ements du r\'eseau. 
Dans le domaine \'energ\'etique, nous avons trouv\'e un brevet qui reconstruit des sous-graphes du r\'eseau \`a partir de la propagation des incidents.
Tous ces travaux supposent que nous connaissons l'\'etat de tous les \'el\'ements du r\'eseau. Ce qui est contraire \`a notre probl\'ematique.
\newline
 Nous avons propos\'e deux approches pour r\'esoudre notre probl\`eme. La premi\`ere approche consiste \`a d\'eterminer la corr\'elation entre les arcs \`a partir des mesures et des r\`egles de flots puis \`a construire une matrice de corr\'elation. La seconde approche d\'ecouvre le r\'eseau en effectuant certaines transformations sur la matrice de corr\'elation.  
