Notre probl\`eme est de d\'ecouvrir la topologie d'un r\'eseau \'electrique dont on ignore les sommets et on ne connait que les flots dans chaque arc.
% --------- modelisation probleme ===> Donnees
%\subsection{Donn\'ees}
\begin{enumerate}
\item {\bf Donn\'ees} : 
\newline
Nous  avons donc
\begin{itemize}
	\item Un ensemble d'arcs distincts $2$ \`a $2$ du graphe $G$ dont les extr\'emit\'es \textbf{initiales} et \textbf{finales} sont inconnues.\\ $ A =\{ a_{1}, ... , a_{m} \} $ 
		
	\item \`A chaque arc sont associ\'ees des s\'eries de mesures $M(a_{i})$. \\
		$\forall a \in A$, $M(a) = ( gp_{a}^{x})_{x\in GP^{a}}$
		
	\item Chaque s\'erie de mesures $ gp_{a}^{x}$ est de norme $T^{a, x}$ v\'erifiant la remarque de la section ~\ref{remarque},  $x \in GP^{a} \subset GP$
	
	\item $\forall t < T^{a, x}$, $  gp_{a}^{x}[t] $ respecte les r\`egles dans $R$.
\end{itemize}
% --------- modelisation probleme ===> objectif
%\subsection{Objectif}
\item {\bf Objectif} : 
\newline
D\'eterminer la topologie  du graphe $G$ \`a partir des flots des arcs $M(a_{i})$ et des r\`egles $R$ c'est-\`a-dire $\forall a_i$, d\'eterminer $n_1,n_2$ tels que $a_i = (n_1,n_2)$.

% --------- approche
%\subsection{Approche}
\item {\bf Approche} :
\newline
Notre objectif est de d\'eduire la topologie du r\'eseau \'electrique repr\'esent\'ee sous la forme d'un graphe de flots. Notre approche se subdivise en deux \'etapes :
\begin{enumerate}
	\item La premi\`ere \'etape consiste \`a la recherche d'arcs ayant des extr\'emit\'es communes. Pour ce faire, nous calculons la similarit\'e entre les paires de mesures d'arcs pour chaque grandeur dans le but  de d\'eterminer les arcs corr\'el\'es puis nous construirons la matrice de corr\'elation. 
	\item La seconde \'etape est la construction du graphe \`a partir de la matrice de corr\'elation. 
\end{enumerate}


\end{enumerate}