\thispagestyle{empty}

%\color{bordeau} \hfill \vfill \tiny \ecodocnum
\begin{textblock}{5}(0,0)
	\textblockcolour{bordeau}
	%\vspace{10mm}
	\includegraphics [scale=0.9]{media/bande.png}
	\vspace{300mm}
\end{textblock}



\begin{textblock}{1}(0.3,3)
	\Large{\rotatebox{90}{\color{white}{NNT : \NNT}}}
\end{textblock}


                            

\begin{textblock}{1}(\hpostt,\vpostt)
	\textblockcolour{white}
	\includegraphics[scale=1]{media/etab/\logoEtt.png}
\end{textblock}

\begin{textblock}{1}(\hpos,\vpos)
	\textblockcolour{white}
		\includegraphics[scale=1]{media/etab/\logoEt.png}	
\end{textblock}

%\vspace{6cm}
%% Texte
\begin{textblock}{10.3}(5.4,3)
	\textblockcolour{white}
	
	\color{bordeau}
	%\begin{center}  
	\begin{flushright}
		\huge{\PhDTitle} \bigskip %% Titre de la thèse 
		\vfill
		\color{black} %% Couleur noire du reste du texte
		\normalsize {Th\`ese de doctorat de l'Universit\'e Paris-Saclay} \\
		pr\'epar\'ee \`a \PhDworkingplace \\ \bigskip
		\vfill
		Ecole doctorale n$^{\circ}$\ecodocnum ~\ecodoctitle ~(\ecodocacro)  \\
		
		\small{Sp\'ecialit\'e de doctorat: \PhDspeciality} \bigskip %% Spécialité 
		\vfill  
		\footnotesize{Th\`ese pr\'esent\'ee et soutenue \`a \defenseplace, le \defensedate, par} \bigskip
		\vfill
		\Large{\textbf{\textsc{\PhDname}}} %% Nom du docteur
		\vfill
		%\bigskip
	\end{flushright}
	
	%\end{center}
	\color{black}
	%% Jury
	\begin{flushleft}
		
		\small Composition du Jury :
	\end{flushleft}
	%% Members of the jury

	\small
	%\begin{center}
	\newcolumntype{L}[1]{>{\raggedright\let\newline\\\arraybackslash\hspace{0pt}}m{#1}}
	\newcolumntype{R}[1]{>{\raggedleft\let\newline\\\arraybackslash\hspace{0pt}}lm{#1}}
	
	\label{jury} 																				%% Mettre à jour si des membres ont été ajoutés ou retirés / Update if members have been added or removed
	\begin{flushleft}
	\begin{tabular}{@{} L{10cm} R{4cm}}
		\jurynameA  \\ \juryadressA & \juryroleA \\
		\jurynameB  \\ \juryadressB & \juryroleB \\
		\jurynameC  \\ \juryadressC & \juryroleC \\
		\jurynameD  \\ \juryadressD & \juryroleD \\
		\jurynameE  \\ \juryadressE & \juryroleE \\
		\jurynameF  \\ \juryadressF & \juryroleF \\
		\jurynameG  \\ \juryadressG & \juryroleG \\
		\jurynameH  \\ \juryadressH & \juryroleH \\
	\end{tabular} 
	\end{flushleft}   
	%\end{center}
\end{textblock}
